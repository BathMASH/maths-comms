% Options for packages loaded elsewhere
\PassOptionsToPackage{unicode}{hyperref}
\PassOptionsToPackage{hyphens}{url}
%
\documentclass[
]{book}
\usepackage{amsmath,amssymb}
\usepackage{lmodern}
\usepackage{iftex}
\ifPDFTeX
  \usepackage[T1]{fontenc}
  \usepackage[utf8]{inputenc}
  \usepackage{textcomp} % provide euro and other symbols
\else % if luatex or xetex
  \usepackage{unicode-math}
  \defaultfontfeatures{Scale=MatchLowercase}
  \defaultfontfeatures[\rmfamily]{Ligatures=TeX,Scale=1}
\fi
% Use upquote if available, for straight quotes in verbatim environments
\IfFileExists{upquote.sty}{\usepackage{upquote}}{}
\IfFileExists{microtype.sty}{% use microtype if available
  \usepackage[]{microtype}
  \UseMicrotypeSet[protrusion]{basicmath} % disable protrusion for tt fonts
}{}
\makeatletter
\@ifundefined{KOMAClassName}{% if non-KOMA class
  \IfFileExists{parskip.sty}{%
    \usepackage{parskip}
  }{% else
    \setlength{\parindent}{0pt}
    \setlength{\parskip}{6pt plus 2pt minus 1pt}}
}{% if KOMA class
  \KOMAoptions{parskip=half}}
\makeatother
\usepackage{xcolor}
\usepackage{longtable,booktabs,array}
\usepackage{calc} % for calculating minipage widths
% Correct order of tables after \paragraph or \subparagraph
\usepackage{etoolbox}
\makeatletter
\patchcmd\longtable{\par}{\if@noskipsec\mbox{}\fi\par}{}{}
\makeatother
% Allow footnotes in longtable head/foot
\IfFileExists{footnotehyper.sty}{\usepackage{footnotehyper}}{\usepackage{footnote}}
\makesavenoteenv{longtable}
\usepackage{graphicx}
\makeatletter
\def\maxwidth{\ifdim\Gin@nat@width>\linewidth\linewidth\else\Gin@nat@width\fi}
\def\maxheight{\ifdim\Gin@nat@height>\textheight\textheight\else\Gin@nat@height\fi}
\makeatother
% Scale images if necessary, so that they will not overflow the page
% margins by default, and it is still possible to overwrite the defaults
% using explicit options in \includegraphics[width, height, ...]{}
\setkeys{Gin}{width=\maxwidth,height=\maxheight,keepaspectratio}
% Set default figure placement to htbp
\makeatletter
\def\fps@figure{htbp}
\makeatother
\setlength{\emergencystretch}{3em} % prevent overfull lines
\providecommand{\tightlist}{%
  \setlength{\itemsep}{0pt}\setlength{\parskip}{0pt}}
\setcounter{secnumdepth}{5}
\usepackage{cancel}

\usepackage{color}
\usepackage{framed}
\setlength{\fboxsep}{.8em}

\newenvironment{information}{
  \definecolor{shadecolor}{rgb}{0.8, 0.9,1.0}
  \definecolor{text-colour}{rgb}{0, 0.25, 0.52}
  \color{text-colour}
  \begin{shaded}}
 {\end{shaded}}
\ifLuaTeX
  \usepackage{selnolig}  % disable illegal ligatures
\fi
\IfFileExists{bookmark.sty}{\usepackage{bookmark}}{\usepackage{hyperref}}
\IfFileExists{xurl.sty}{\usepackage{xurl}}{} % add URL line breaks if available
\urlstyle{same} % disable monospaced font for URLs
\hypersetup{
  pdftitle={Communicating Maths Information},
  pdfauthor={Tamsin Smith},
  hidelinks,
  pdfcreator={LaTeX via pandoc}}

\title{Communicating Maths Information}
\author{Tamsin Smith}
\date{2023-02-14}

\begin{document}
\maketitle

{
\setcounter{tocdepth}{1}
\tableofcontents
}
\hypertarget{welcome}{%
\chapter*{Welcome}\label{welcome}}
\addcontentsline{toc}{chapter}{Welcome}

This document contains an overview of important dates and details for communicating maths students 2023. This document will update as the semester progresses.

\hypertarget{general-course-overview}{%
\chapter*{General Course Overview}\label{general-course-overview}}
\addcontentsline{toc}{chapter}{General Course Overview}

General Course Overview

Week Begining

Focus

6 Feb 23

Input from staff

13 Feb 23

20 Feb 23

27 Feb 23

6 Mar 23

Output from students

13 mar 23

20 Mar 23

27 Mar 23

3 Apr 23

Easter break

10 Apr 23

17 Apr 23

Mentoring

24 Apr 23

Hand in this week

1 May 23

\hypertarget{week-by-week-detail}{%
\chapter*{Week by week detail}\label{week-by-week-detail}}
\addcontentsline{toc}{chapter}{Week by week detail}

Lectures,workshops and mentoring sessions will take place at the following times in weeks 19-26,29,30.

\begin{itemize}
\item
  \textbf{Thursday} 12:15-13:05 in 8W2.30 †
\item
  \textbf{Friday} 12:15-14:05 in 1W2.104
\end{itemize}

† please note that in some of the Thursday sessions, the cohort will split into three different locations, each group with their own mentor. Details of this will be listed in the week summary below.

Week 1 WB 6/2/23, click to reveal.

\begin{information}
\textbf{Thursday 9/2/2023, 12:15-13:05 in 8W2.30}

Welcome.

What is Maths Communication?

Course information.

\end{information}

\(~\)

\begin{information}
\textbf{Friday 10/2/2023, 12:15-14:05 in 1W2.104 }

Exploring the existing maths communication out there.

Setting up the `Compare and Contrast' formative assessment.

\end{information}

Week 2, WB 13/2/23, click to reveal.

\begin{information}
\textbf{Thursday 16/2/2023, 12:15-13:05 in 8W2.30}

`Compare and Contrast' formative assessment discussions.

\end{information}

\(~\)

\begin{information}
\textbf{Friday 17/2/2023, 12:15-14:05 in 1W2.104 }

Masterclasses overview.

Explore what a Masterclass is, tips on how to plan one and time to start your initial discussions.

\end{information}

Week 3 WB 20/2/23, click to reveal.

\begin{information}
\textbf{Thursday 23/2/2023, 12:15-13:05 in 8W2.30}

Aims and Objectives.

What are they? Examples in practice and their benefit for evaluation.

\end{information}

\(~\)

\begin{information}
\textbf{Friday 24/2/2023, 12:15-14:05 in 1W2.104 }

The greatest hits of maths communication.

We will showcase a selection of the most used examples in the world of maths comm as inspiration for content choice and delivery options for your own delivery.

\end{information}

Week 4 WB 27/2/23, click to reveal.

\begin{mentoring}
\textbf{Thursday 2/3/2023, 12:15-13:05 in 8W2.30}

MENTORING

Details of where and when will appear here!

\end{mentoring}

\(~\)

\begin{information}
\textbf{Friday 3/3/2023, 12:15-14:05 in 1W2.104 }

Presentation skills and managing an audience

We will provide examples of a variety of presenter/audience interactions and highlight successful, and potential likely areas of breakdown, in this interaction.

\end{information}

\hypertarget{assessment-information}{%
\chapter*{Assessment information}\label{assessment-information}}
\addcontentsline{toc}{chapter}{Assessment information}

Assessment credit breakdown.

The Assessment of the course will be based around two Mathematical Interactions (MI's).

A \textbf{Mathematical Interaction} (MI) is defined as the communication of mathematics to an audience.

Typically:

\begin{itemize}
\item
  MI1 is an Ri Masterclass, planned and delivered as a group.
\item
  MI2 is an option from video, article or school placement, planned and delivered as an individual or pair.
\end{itemize}

Each of the two MI's will be summatively assessed in two parts:

\begin{enumerate}
\def\labelenumi{\Alph{enumi})}
\item
  The live And recorded delivery; 25\% of the overall course mark
\item
  The written evaluation; 25\% of the overall course mark
\end{enumerate}

Overall this is how the marks for the course will be allocated:

Delivery

Evaluation report

Mathematical Interaction 1

25\%

25\%

Mathematical Interaction 2

25\%

25\%

The assessment criteria for each part can be found in Assessment submission topic file on moodle or in the list below titled Assessment Criteria.

Coursework cover sheet.

\textbf{Set:} When issued: 9.2.23 12:15.

\textbf{Due:} Deadline: 28.4.23 11:59pm.

\textbf{Estimated time required:} The coursework should take no more than \emph{30 hours} to complete. This is an upper bound and most students will complete the written report in less time.

\textbf{Submission:} Submit to the moodle page as a PDF; Your candidate name and number be included.

\textbf{Conditions:} The report is an individual submission.

\textbf{Value:} This written assignment carries 50\% of the total marks for the course.This assignment will be marked out of 50 where 25 marks are available for each section detailed on the assessment criteria.

\textbf{Length:} There is a word count of 2500 words, 1250 words for your evaluation of Mathematical Interaction 1 and 1250 word for your evaluation of Mathemametical Interaction 2.

\textbf{Support and advice:} You can ask you mentor for advice up until the deadline day in the designated mentoring sessions.

\textbf{Feedback:} You will receive your marks for the unit at the same time as your other modules.

\textbf{Late submission of coursework:} If there are valid circumstances preventing you from meeting the deadline, your Director of Studies may grant you an extension to the specified submission date, if it is requested before the deadline. Forms to request an extension are available on SAMIS.

\begin{itemize}
\tightlist
\item
  If you submit a piece of work after the submission date, and no extension has been granted, the maximum mark possible will be the pass mark.
\item
  If you submit work more than five working days after the submission date, you will normally receive a mark of 0 (zero), unless you have been granted an extension.
\end{itemize}

\textbf{Academic integrity statement:} Academic misconduct is defined by the University as ``the use of unfair means in any examination or assessment procedure''. This includes (but is not limited to) cheating, collusion, plagiarism, fabrication, or falsification. The University's Quality Assurance Code of Practice, \href{https://www.bath.ac.uk/publications/qa53-examination-and-assessment-offences/}{QA53 Examination and Assessment Offences}, sets out the consequences of committing an offence and the penalties that might be applied.

\textbf{Contact details:}

\emph{Tamsin Smith}\newline
Room: \emph{6W1.28} \newline
E-mail: \emph{\href{mailto:t.l.smith@bath.ac.uk}{\nolinkurl{t.l.smith@bath.ac.uk}}}

Group Contributions.

\emph{Group Contribution Rating (GCR)}

For the elements of the course where you will present material as a group, group members must agree on one another's relative contribution to their coursework. We will assume that groups agree to an even allocation of marks across the group unless a Group Contribution Form (GCF) is submitted to the unit convener.

\emph{The Group Contribution Form (GCF)}

The GCF is available on the Moodle page and records how well individual members have contributed to the coursework throughout the coursework period. The final GCF is a one-page document with every group member's name, signature and agreed assessment of their percentage contribution to the coursework over the whole coursework period. In the case that recorded percentage contributions are exceptionally high or low, the GCF document should include a brief paragraph of explanation. (An example can be found Assessment submission topic file). The signed GCF must be submitted to the unit convener after the live delivery. If no agreement can be reached on the contributions of individual members, the group can hand in more than one GCF, but members will be required to produce evidence in a hearing with the unit convener to support their argument. In such an eventuality, the Director of Teaching may also be involved. The GCF will directly inform the allocation of marks to each individual group member.

Assessment Criteria.

\textbf{Live and recorded delivery descriptors.}

\emph{Grid A} will be used to mark the content of each Mathematical Interaction (which could be one of the below items):

\begin{itemize}
\tightlist
\item
  COMPULSORY - Masterclasses: Live delivery - in groups
  \(~\)
\item
  OPTION - Written article
  suitable for Chalkdust, published on Medium - individual work
\item
  OPTION - Video
  published on YouTube, e.g.~similar in style to Numberphile, StandupMaths, 3Blue1Brown. Maximum 2 in group.
\item
  OPTION - School placement
  to be arranged in local school (limited availability)
\end{itemize}

\textbf{Written Evaluation descriptors}

\emph{Grid B} will be used to mark the written evaluation of each Mathematical Interaction.

See below for Grid A and B, please note that a one page pdf document of these tables is available on moodle.

\emph{Grid A}

\begin{longtable}[]{@{}
  >{\raggedright\arraybackslash}p{(\columnwidth - 10\tabcolsep) * \real{0.0150}}
  >{\raggedright\arraybackslash}p{(\columnwidth - 10\tabcolsep) * \real{0.1850}}
  >{\raggedright\arraybackslash}p{(\columnwidth - 10\tabcolsep) * \real{0.1549}}
  >{\raggedright\arraybackslash}p{(\columnwidth - 10\tabcolsep) * \real{0.1873}}
  >{\raggedright\arraybackslash}p{(\columnwidth - 10\tabcolsep) * \real{0.1931}}
  >{\raggedright\arraybackslash}p{(\columnwidth - 10\tabcolsep) * \real{0.2647}}@{}}
\toprule()
\begin{minipage}[b]{\linewidth}\raggedright
Mark
\end{minipage} & \begin{minipage}[b]{\linewidth}\raggedright
Mathematical Content (e.g.~equations, derivations, proofs and correctness/appropriateness of these)
\end{minipage} & \begin{minipage}[b]{\linewidth}\raggedright
Communication with media and demonstration (e.g.~slides, physical props, board use, graphics/animation, handouts etc)
\end{minipage} & \begin{minipage}[b]{\linewidth}\raggedright
Style and Delivery (e.g technique, audience handling, confidence, clarity, surprise, humour, variety, tone)
\end{minipage} & \begin{minipage}[b]{\linewidth}\raggedright
Structure and pacing (e.g sections, pace, flow, narrative, coherence)
\end{minipage} & \begin{minipage}[b]{\linewidth}\raggedright
Appropriateness for Audience
\end{minipage} \\
\midrule()
\endhead
5 & Detailed and consistent understanding of the content. Fully correct and appropriate mathematics. Awareness of wider context demonstrated. & Fully developed use of media. Successful, creative, or original demonstrations. & Well developed general style and delivery. Delivery is clear, and engages well with the audience, using a range of successful techniques. & Fully appropriate structure and pace. Well chosen quantity and ordering of material gives a complete narrative. & Fully developed understanding of the intended audience and their prior knowledge. Progression of the material, from an appropriate entry level, is successfully managed and enhances audience interest/experience. \\
4 & Detailed understanding of the content. Mostly correct and appropriate mathematics. & Mostly developed use of media. Mostly successful, creative, or original demonstrations. & General style and delivery is well developed in places. Delivery engages well with the audience, using some successful techniques. & Appropriate structure and pace. Quantity and order of material fits structure and aids narrative. & Mostly developed understanding of the intended audience and their prior knowledge. Progression of the material achieved, from an appropriate entry level. \\
3 & Some misunderstanding of the content is apparent. Some mathematical inaccuracies or omissions. & Some media used inappropriately, otherwise resources and media are adequate. Some successful demonstrations. & Some good style and delivery. Occasional problems with clarity or audience engagement. & Some evidence of structure and some sections well paced. Some problems with quantity or ordering of material which affect the narrative. & Some understanding of the intended audience and their prior knowledge. Some progression of material is successfully managed. Some material occasionally inappropriate for the audience. \\
2 & Significant misunderstanding of the content is apparent. Significant mathematical inaccuracies or omissions. & Significant problems with use of media, even if some are used well. Significant problems with demonstrations. & Significant problems with style or delivery, with negative consequences for audience engagement (observed or likely). & Minimal structure apparent or significant problems with pacing. Significant problems with quantity or ordering of material which affect the narrative. & Minimal understanding of the audience and their prior knowledge. Significant problems with level of material and progression. \\
1 & Very limited understanding of content Minimal mathematical content. & Very limited use of media. Very limited demonstrations. & Very limited style and delivery throughout, resulting in continued audience disengagement (observed or likely). & Very limited or no structure Very limited pacing. No overarching narrative. & Very limited understanding of the audience. Inappropriate level and progression of material. \\
0 & No mathematical content. & No use of media or demonstrations. & No presentation or work submitted. & No presentation or work submitted. & No indication or awareness of the intended audience or their level. \\
\bottomrule()
\end{longtable}

\emph{Grid B}

\begin{longtable}[]{@{}
  >{\raggedright\arraybackslash}p{(\columnwidth - 10\tabcolsep) * \real{0.0066}}
  >{\raggedright\arraybackslash}p{(\columnwidth - 10\tabcolsep) * \real{0.2029}}
  >{\raggedright\arraybackslash}p{(\columnwidth - 10\tabcolsep) * \real{0.1996}}
  >{\raggedright\arraybackslash}p{(\columnwidth - 10\tabcolsep) * \real{0.2370}}
  >{\raggedright\arraybackslash}p{(\columnwidth - 10\tabcolsep) * \real{0.2591}}
  >{\raggedright\arraybackslash}p{(\columnwidth - 10\tabcolsep) * \real{0.0948}}@{}}
\toprule()
\begin{minipage}[b]{\linewidth}\raggedright
Mark
\end{minipage} & \begin{minipage}[b]{\linewidth}\raggedright
Aims, Objectives and Intentions
\end{minipage} & \begin{minipage}[b]{\linewidth}\raggedright
Collection and analysis of Feedback
\end{minipage} & \begin{minipage}[b]{\linewidth}\raggedright
Positive reflections
\end{minipage} & \begin{minipage}[b]{\linewidth}\raggedright
Potential Improvements
\end{minipage} & \begin{minipage}[b]{\linewidth}\raggedright
Overall quality of written communication
\end{minipage} \\
\midrule()
\endhead
5 & Fully appropriate aims and objectives that are clear, attainable, and appropriate. Fully developed discussion of intentions with reasons, and the specific actions to accomplish them. & Fully developed plan for collecting appropriate feedback from the audience. Detailed analysis of the quantity and quality of feedback received using meaningful summary techniques. & Fully developed discussion of what went well, and why. Clear and detailed reflection on success of meeting the overall aims and objectives. Characterised by mature reflective judgement on the impact of the work. & Fully developed discussion of what did not work well, and how things could be improved. Clear reflection on failure of meeting the overall aims and objectives. Characterised by mature reflective judgement on the impact of the work. & Ideas expressed clearly, concisely, and with excellent structure. \\
4 & Appropriate aims and objectives that are mostly clear, attainable, and appropriate. Appropriate discussion of intentions. & Developed methods of feedback collection feedback from the audience. Analysis of the quantity or quality of feedback received using meaningful summary techniques. & Developed discussion of what went well, and why. Clear reflection on success of meeting the overall aims and objectives. & Developed discussion of what did not work well. Clear reflection on failures of meeting the overall aims and objectives. & Ideas are generally expressed well, and with appropriate structure. \\
3 & Fairly clear aims and objectives some of which are clear, attainable, and appropriate. Partial discussion of intentions/reasons. & Some evidence of feedback collected, but of varying quality and usefulness. Some discussion/analysis, possibly presented inefficiently. & Some awareness of what went well and only partial discussion of the reasons for this. & Some awareness of what did not work well and only partial discussion of the reasons for this. & Ideas are generally expressed fairly clearly, with reasonably appropriate structure. \\
2 & Mostly unhelpful aims and objectives, which are not clear, obtainable or appropriate. Minimal discussion of intentions/reasons. & Minimal planning for feedback leading to little discussion or analysis. & Minimal awareness of what went well. Obvious success points missed or not discussed. & Minimal awareness of what did not work well. Obvious failure points missed or not discussed & Ideas are rarely expressed coherently. Structure is less appropriate. \\
1 & Aims or objectives are unhelpful or vague, and few or none are clear or obtainable. Very limited discussion of intentions/reasons. & Very limited planning for feedback collection and little or no discussion or analysis. & Very limited awareness/reflection on any success. & Very limited awareness/reflection on any failure. & Ideas are expressed incoherently. Limited structure. \\
0 & No aims or objectives set. & No attempt at feedback collection, discussion, or analysis. & No awareness of positive aspects of work. & No awareness of negative aspects of work. & No written work submitted. \\
\bottomrule()
\end{longtable}

\hypertarget{masterclasses}{%
\chapter*{Masterclasses}\label{masterclasses}}
\addcontentsline{toc}{chapter}{Masterclasses}

The Communicating Maths cohort will deliver 5 masterclasses over two weekends.

Please find your group below to find details of your delivery group and the series you will attend.

Bath 18.3.23.

\begin{information}
\textbf{Date:} \emph{18.3.23}.

\textbf{Time:} \emph{10:00-12:30}.

\textbf{Location:} \emph{Bath University Campus}.

\textbf{Audience:} \emph{Year 9, approx 100 participants}.

\textbf{University Mentor:} \emph{Tam}.

\textbf{Masterclass Contact} \emph{Chris Budd \href{mailto:mascjb@bath.ac.uk}{\nolinkurl{mascjb@bath.ac.uk}}}.

\textbf{Group Members:} \emph{Joy Boh, Ashley Higgs, Chloe Howcroft, Lucia Lopez, Haranja Sivaneswaran, Jodie Young.}

\textbf{Initial pitch:} This group will pitch their initial ideas for their masterclass to the rest of the group at 12:20 on 9/3/23. You should prepare a 10 minute overview of your masterclass showing any relevant material. After this, you can expect to receive questions and feedback from your lecturers and peers for approx 10 minutes.

\textbf{Feedback:} After you have delivered your Masterclass you will receive verbal feedback, to support your written evaluation. This will be given by your mentor, who observed. Feedback slots will be approx 20 minutes long and during the 12:15-13:05 timetabled time on 30/3/23. You will be expected to ask your mentor reflective questions to support your evaluation.

\end{information}

Oxford 18.3.23.

\begin{information}
\textbf{Date:} \emph{18.3.23}.

\textbf{Time:} \emph{10:00-12:30}.

\textbf{Location:} \emph{Oxford Brookes Campus}.

\textbf{Audience:} \emph{Sixth form, approx 30 participants}.

\textbf{University Mentor:} \emph{Ben}.

\textbf{Masterclass Contact} \emph{Sam Kamparis \href{mailto:s.kamperis@brookes.ac.uk}{\nolinkurl{s.kamperis@brookes.ac.uk}}}.

\textbf{Group Members:} \emph{Tom Eves, Zoe Fairfax, Alankrit Mata, Anna Quinn, Rhea Shah.}

\textbf{Initial pitch:} This group will pitch their initial ideas for their masterclass to the rest of the group at 12:20 on 10/3/23. You should prepare a 10 minute overview of your masterclass showing any relevant material. After this, you can expect to receive questions and feedback from your lecturers and peers for approx 10 minutes.

\textbf{Feedback:} After you have delivered your Masterclass you will receive verbal feedback, to support your written evaluation. This will be given by your mentor, who observed. Feedback slots will be approx 20 minutes long and during the 12:15-13:05 timetabled time on 30/3/23. You will be expected to ask your mentor reflective questions to support your evaluation.

\end{information}

Holyport College 18.3.23.

\begin{information}
\textbf{Date:} \emph{18.3.23}.

\textbf{Time:} \emph{9:30-12:00}.

\textbf{Location:} \emph{Holyport College}.

\textbf{Audience:} \emph{Year 9, approx 45 participants}.

\textbf{University Mentor:} \emph{Waleed}.

\textbf{Masterclass Contact} \emph{Daniel Hubbard \href{mailto:rimasterclasses@holyportcollege.org.uk}{\nolinkurl{rimasterclasses@holyportcollege.org.uk}}}.

\textbf{Group Members:} \emph{Andrew McGrath ,Kamilla Bugno,Ewan Partington, Samantha Perryman, Olivia Wiseman.}

\textbf{Initial pitch:} This group will pitch their initial ideas for their masterclass to the rest of the group at 12:40 on 9/3/23. You should prepare a 10 minute overview of your masterclass showing any relevant material. After this, you can expect to receive questions and feedback from your lecturers and peers for approx 10 minutes.

\textbf{Feedback:} After you have delivered your Masterclass you will receive verbal feedback, to support your written evaluation. This will be given by your mentor, who observed. Feedback slots will be approx 20 minutes long and during the 12:15-13:05 timetabled time on 30/3/23. You will be expected to ask your mentor reflective questions to support your evaluation.

\end{information}

Bath 25.3.23.

\begin{information}
\textbf{Date:} \emph{25.3.23}.

\textbf{Time:} \emph{10:00-12:30}.

\textbf{Location:} \emph{Bath University Campus}.

\textbf{Audience:} \emph{Year 9, approx 100 participants}.

\textbf{University Mentor:} \emph{Ben}.

\textbf{Masterclass Contact} \emph{Chris Budd \href{mailto:mascjb@bath.ac.uk}{\nolinkurl{mascjb@bath.ac.uk}}}.

\textbf{Group Members:} \emph{Abigail Bennett, Tosia Ciszek, Mirren Derby,Timi Folaranmi, Molly Maguire-King, Tali Shear}

\textbf{Initial pitch:} This group will pitch their initial ideas for their masterclass to the rest of the group at 12:40 on 10/3/23. You should prepare a 10 minute overview of your masterclass showing any relevant material. After this, you can expect to receive questions and feedback from your lecturers and peers for approx 10 minutes.

\textbf{Feedback:} After you have delivered your Masterclass you will receive verbal feedback, to support your written evaluation. This will be given by your mentor, who observed. Feedback slots will be approx 20 minutes long and during the 12:15-13:05 timetabled time on 30/3/23. You will be expected to ask your mentor reflective questions to support your evaluation.

\end{information}

Exeter 25.3.23.

\begin{information}
\textbf{Date:} \emph{25.3.23}.

\textbf{Time:} \emph{9:30-12:00}.

\textbf{Location:} \emph{Exeter School of Maths}.

\textbf{Audience:} \emph{Year 9, approx 30 participants}.

\textbf{University Mentor:} \emph{Tam}.

\textbf{Masterclass Contact} \emph{Sophie Brown \href{mailto:SophieBrown@exeterms.ac.uk}{\nolinkurl{SophieBrown@exeterms.ac.uk}}}.

\textbf{Group Members:} \emph{Sam Cheung, Lana Gregory, Ella Hidveghy, Rebecca Knight, Mira Balaji.}

\textbf{Initial pitch:} This group will pitch their initial ideas for their masterclass to the rest of the group at 13:20 on 10/3/23. You should prepare a 10 minute overview of your masterclass showing any relevant material. After this, you can expect to receive questions and feedback from your lecturers and peers for approx 10 minutes.

\textbf{Feedback:} After you have delivered your Masterclass you will receive verbal feedback, to support your written evaluation. This will be given by your mentor, who observed. Feedback slots will be approx 20 minutes long and during the 12:15-13:05 timetabled time on 30/3/23. You will be expected to ask your mentor reflective questions to support your evaluation.

\end{information}

\hypertarget{school-placements}{%
\chapter*{School Placements}\label{school-placements}}
\addcontentsline{toc}{chapter}{School Placements}

A list of the opportunities that you can opt into for your second Mathematical Interaction will appear below.

\emph{Please note that you need to have opted to do this, by emailing \href{mailto:t.l.smith@bath.ac.uk}{\nolinkurl{t.l.smith@bath.ac.uk}} by 11:15 am on 17.2.23}

\begin{information}
\textbf{Oldfield School \(i^2\) club}

\textbf{Date:} 28.3.23.

\textbf{Time:} 13:10-13:50.

\textbf{Location:} \href{http://www.oldfieldschool.com/}{Oldfield School}

\textbf{Age range:} Year 11,12,13, ages 15-18.

\textbf{Numbers expected:} 10-15.

\textbf{Ability expected:} Mixed, This is a voluntary club so anyone can attend, however usually this is a club that high attaining pupils attend.

\textbf{Type of placement} \(i^2\) is a voluntary enrichment club for interested students at a fully comprehensive, mixed sex school.

\textbf{Notes:} You need be able to make your own way to the venue.

\end{information}

\(~\)

\begin{information}
\textbf{Oldfield School enrichment lesson.}

\textbf{Date:} 23.3.23

\textbf{Time:} 8:50-9:50.

\textbf{Location:} \href{http://www.oldfieldschool.com/}{Oldfield School}

\textbf{Age range:} Year 10, ages 14-15.

\textbf{Numbers expected:} 32.

\textbf{Ability expected:} This is a top set class.

\textbf{Type of placement} This is an opportunity to take a one off lesson to a group of high ability students in a fully comprehensive, mixed sex school.

\textbf{Notes:} You need be able to make your own way to the venue.

\end{information}

\(~\)

\begin{information}
\textbf{Oldfield School enrichment lesson.}

\textbf{Date:} 24.3.23

\textbf{Time:} 9:50-10:50.

\textbf{Location:} \href{http://www.oldfieldschool.com/}{Oldfield School}

\textbf{Age range:} Year 9, ages 13-14.

\textbf{Numbers expected:} 32.

\textbf{Ability expected:} This is a top set class.

\textbf{Type of placement} This is an opportunity to take a one off lesson to a group of high ability students in a fully comprehensive, mixed sex school.

\textbf{Notes:} You need be able to make your own way to the venue.

\end{information}

\(~\)

\begin{information}
\textbf{Oldfield School enrichment lesson.}

\textbf{Date:} 27.3.23

\textbf{Time:} 9:50-10:50.

\textbf{Location:} \href{http://www.oldfieldschool.com/}{Oldfield School}

\textbf{Age range:} Year 9, ages 13-14.

\textbf{Numbers expected:} 32.

\textbf{Ability expected:} This is a top set class.

\textbf{Type of placement} This is an opportunity to take a one off lesson to a group of high ability students in a fully comprehensive, mixed sex school.

\textbf{Notes:} You need be able to make your own way to the venue.

\end{information}

\(~\)

\begin{information}
\textbf{Oldfield School enrichment lesson.}

\textbf{Date:} 27.3.23

\textbf{Time:} 11:10-12:10.

\textbf{Location:} \href{http://www.oldfieldschool.com/}{Oldfield School}

\textbf{Age range:} Year 8, ages 12-13.

\textbf{Numbers expected:} 32.

\textbf{Ability expected:} This is a top set class.

\textbf{Type of placement} This is an opportunity to take a one off lesson to a group of high ability students in a fully comprehensive, mixed sex school.

\textbf{Notes:} You need be able to make your own way to the venue.

\end{information}

\(~\)

\begin{information}
\textbf{Oldfield School enrichment lesson.}

\textbf{Date:} 30.3.23

\textbf{Time:} 8:50-9:50.

\textbf{Location:} \href{http://www.oldfieldschool.com/}{Oldfield School}

\textbf{Age range:} Year 8, ages 12-13.

\textbf{Numbers expected:} 32.

\textbf{Ability expected:} This is a top set class.

\textbf{Type of placement} This is an opportunity to take a one off lesson to a group of high ability students in a fully comprehensive, mixed sex school.

\textbf{Notes:} You need be able to make your own way to the venue.

\end{information}

\(~\)

\begin{information}
\textbf{Oldfield School enrichment lesson.}

\textbf{Date:} 30.3.23

\textbf{Time:} 9:50-10:50.

\textbf{Location:} \href{http://www.oldfieldschool.com/}{Oldfield School}

\textbf{Age range:} Year 7, ages 11-12.

\textbf{Numbers expected:} 32.

\textbf{Ability expected:} This is a top set class.

\textbf{Type of placement} This is an opportunity to take a one off lesson to a group of high ability students in a fully comprehensive, mixed sex school.

\textbf{Notes:} You need be able to make your own way to the venue.

\end{information}

\(~\)

\begin{information}
\textbf{Oldfield School enrichment lesson.}

\textbf{Date:} 31.3.23

\textbf{Time:} 8:50-9:50.

\textbf{Location:} \href{http://www.oldfieldschool.com/}{Oldfield School}

\textbf{Age range:} Year 8, ages 12-13.

\textbf{Numbers expected:} 32.

\textbf{Ability expected:} This is a top set class.

\textbf{Type of placement} This is an opportunity to take a one off lesson to a group of high ability students in a fully comprehensive, mixed sex school.

\textbf{Notes:} You need be able to make your own way to the venue.

\end{information}

\(~\)

\begin{information}
\textbf{Oldfield School enrichment lesson.}

\textbf{Date:} 31.3.23

\textbf{Time:} 9:50-10:50.

\textbf{Location:} \href{http://www.oldfieldschool.com/}{Oldfield School}

\textbf{Age range:} Year 10, ages 14-15.

\textbf{Numbers expected:} 32.

\textbf{Ability expected:} This is a top set class.

\textbf{Type of placement} This is an opportunity to take a one off lesson to a group of high ability students in a fully comprehensive, mixed sex school.

\textbf{Notes:} You need be able to make your own way to the venue.

\end{information}

\end{document}
